\chapter*{Anexos}
\label{anexo:refinamentos}
\addcontentsline{toc}{chapter}{ANEXOS}
%Adiciona o capítulo ANEXOS ao sumário

\renewcommand\thesection{A}
%Ao invés de colocar no sumário em números, este renewcommand coloca como A, ou B, ou C... Tem que mudar manualmente!

\section{Código Python para conversão de bases}
\label{Python-Conversão de bases}

O pacote \textit{verbatim} configura tudo como texto corrido.

\begin{verbatim}

#Pacotes do python.
import numpy as np
import numpy.linalg as la

#Define uma lista alat extraindo os valores do arquivo de entrada.
alat = np.genfromtxt('atomic-positions-alat.txt')

#Define as bases.
b1 = np.array([[1.,0.,0.],[0.,1.,0.],[0.,0.,1]]) # alat
b2 = .5 * np.array( [ [1.,-1.,1.], [1.,1.,1.], [-1.,-1.,1.] ] ) 
# crystal, ibrav=7

#Operação de matrizes.
M1 = np.transpose(b1)
M2 = np.transpose(b2)
M2inv = la.inv(M2)

#Cálculo da mudança de base.
T = np.dot(M2inv,M1)

#Cria uma matriz de zeros 3x3.
crystal = np.zeros([len(alat),3])

#Adiciona os valores convertidos a lista crystal após a mudança de base.
for i in range(len(alat)):
crystal[i] = (np.dot(T, alat[i]))
print "%21.9ft%12.9ft\%12.9f" % (crystal[i,0], crystal[i,1], crystal[i,2])
\end{verbatim}